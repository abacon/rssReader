\documentclass{article}
\usepackage{homework}
\usepackage{amsmath}
\begin{document}

\homework{1}{RSS Reflection}{Holly French, Andrew Bacon, Veronica Lynn}{Foo}

During our code review, there were three things we discussed, mostly pertaining to the notion of code readability, especially depending on who our audience is.  

\begin{enumerate}
\item Long, convoluted conditionals: we had a number of long convoluted conditionals that ended up dealing with a lot of edge cases regarding date and time in our RSS parser.  These conditionals helped us determine when to print, but ultimately, they checked a lot of edge cases.  The conditionals were more than one long line long, so they ended up looking like brute-force solutions to our problem.  Instead, we decided to move the conditionals to another function called \emph{isPrintable} and try to clean them up there.
\item Use of the ternary operator: In our code review, we discussed how readable the ternary operator was....................................
\item Useless comments that described code that was already more-or-less self-documenting..................

\end{enumerate}


\end{document}

%%% Local Variables: 
%%% mode: latex
%%% TeX-master: t
%%% End: 
